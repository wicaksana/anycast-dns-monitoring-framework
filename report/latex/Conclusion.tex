\chapter{Conclusions and Future Work}
\label{ch05}
Conclusions drawn from the results of this work is presented  Section \ref{ch05:conclusions}. It serves as the answers to research questions of this thesis as well. Finally, suggestions for future works are provided in Section \ref{ch05:future-works}.
\section{Conclusions}
\label{ch05:conclusions}

%th the recent exhaustion of IPv4, the deployment of IPv6 networks becomes imminent. As the applications of IPv6 is increasing, the service quality running on both protocols should be comparable. As IPv6 is maturing protocol-wise, study of IPv4 and IPv6 catchment areas is important because now the performances are mostly influenced by control-plane decision. Operators may implement different policies for IPv4 and IPv6. By performing IPv4 and IPv6 catchment analysis, operators may get information whether their routing policies have correctly implemented or not.
In Chapter \ref{ch02}, we see that control plane measurement of anycast service can be done by monitoring the address prefixes using BGP routing information provided by BGP speakers.  Among other alternatives, the use of BGP data provided by monitoring projects such as RIS or RouteViews is the preferred approach for this thesis. Then, in the beginning of Chapter \ref{ch03}, we make justification to only use data from RIS due to time and resource constraints, since it provides access to the BGP resources using REST API, instead of directly working with the dump files which requires large resources.

In Section \ref{ch04:evolution}, we show the evolution of Root Servers' IPv4 and IPv6 catchment areas  as the following. Most Root Servers have tendency of increasing convergence level over the time. In general, the convergence level of Root Servers is relatively high, between 50\% to 80\%, with the exception J- and M-Root (below 40\%). Some Root Servers experience sharp increase (A and D-Root),  and some are relatively stagnated (I, C, and L-Root). Particular exception is for F and M-Root which have temporary moment of decreasing. The changes in convergence level is mostly due to the switch of upstream providers for either IPv4 or IPv6. In terms of Root Servers' visibility as seen by VPs, it is the peering policies which determines it, not the amount of instances deployed. The overall average path length itself is close to 4 hops, which in line with result of similar studies in the past. Furthermore, with the exception of A and D-Root, Root Servers seem not to experience major change on their path lengths over the time.

As for the catchment differences itself, as discussed in Section \ref{ch04:differences}, diverging VPs are mostly dominated with VPs with equal path lengths, with J and M-Root (which are the ones with low convergence level) as the notable exception. J-Root is dominated with shorter IPv4, while M-Root is dominated with shorter IPv6 paths. In terms of average path length, the diverging VPs is slightly longer than the ones of dual-stacked VPs. Furthermore, Root Servers with high convergence level (C, I, and K-Root) have the largest differences. For diverging VPs with different path lengths, the average length difference is only ~1 hop. Finally, one factor largely contributing diverging paths is the practice of direct peering for either IPv4 or IPv6, while the other protocol is still using transit ASes. The practice of direct peering itself is much more commonly used in IPv4 than in IPv6, except for K-Root that have large fraction of it for both protocols.

Finally, the features of our visualization tool described in Section \ref{ch04:visualizing} can be used by operator to quickly perform comparison between IPv4 and IPv6 catchment areas. In case of Root Servers with multiple origin ASes or unique penultimate ASes, it can be used as well to detect route leakage and different serving  instances for IPv4 and IPv6. It also provide depiction of the symmetry of catchment areas. Good catchment area is indicated by more or less equal AS path length for all VPs. Catchments with unbalanced tree is determined by noticeable number of VPs suffering long AS path, This indicates that the Root Server should provide better service to them.

%Root Servers experience varying degree of convergence level over the time. IPv4 and IPv6 networks are still expanding. Hence, this is expected. There are some routing events related to the dynamics of convergence level. Convergence level is primarily influenced by the routing policy and upstream provider, not the quantity of the instances itself.

%It is a common knowledge that IPv6 infrastructure is lacking, compared to IPv4. However, as the result of this study shows, some of those that have both IPv4 and IPv6 are surprisingly have better IPv6 connectivity.

%The visualization can be used to investigate BGP misconfiguration. For example, if operator wants to see whether its local instances are not leaked. Since the visualization works at control-plane level, it can be easily modified for any kind of IP anycast service, simply by replacing the Root Server's prefix with the service prefix.

%\textbf{(high-level conclusion, not only applied for DNS, but all anycasted services)}

\iffalse
\textbf{Revisiting the RQ.}
\begin{description}
	\setlength{\itemsep}{1pt}
	\setlength{\parskip}{0pt}
	\item [\textbf{RQ.1}] \textit{How can we measure the control plane of anycast DNS system?}
	\item [\textbf{RQ.2}] \textit{How do IPv4 and IPv6 catchment areas evolve over time?}
	\item [\textbf{RQ.3}] \textit{How different is IPv4 and IPv6 catchment areas?}
	\item [\textbf{RQ.4}] \textit{How to represent the knowledge to the operator?}
\end{description}
\fi

%To answer the main RQ:

%\textbf{RQ: How different is IPv4 and IPv6 catchment areas of DNS Root Servers?}

\section{Future Work}
\label{ch05:future-works}
As discussed in Section \ref{ch02:control-plane}, there are several limitations with the use of public BGP data from measurement projects such as RIS. The biggest concern is that they do not represent the Internet at all with very limited view over the networks, since the collectors are deployed only at few IXPs. Nevertheless, this is the best option available during our work. Immediate improvement to enrich the datasets can be done by using BGP data from RouteViews, as they put collectors at some different locations as RIS. It will provide more complete view of the networks over Root Servers' catchment areas. BMP is the promising alternative, as it allows us to gather all BGP routing data from a BGP speaker (including routes from peers of a peer). However, it requires the router to implement BMP as well, which might require some time to upgrade the routers in IXPs to include such capability. Nevertheless, it seems to be the go-to direction for measurement projects in the future, as already initiated by RouteViews and Caida.

In this thesis, we analyzed BGP data from VPs during the observation time. We take the snapshot of route information for Root Servers' prefixes. In this way, we can draw conclusion about the evolution of the route. However, we only take snapshot once per month. BGP RIB only provides the result of BGP routing calculation, not what triggers the calculation (\textit{i.e.}, the routing events such as prefix withdrawal, announcement, or changes). To study the dynamics of an anycast service from control plane perspective in a finer resolution (\textit{e.g.},route stability), a further study on BGP update messages of the respective prefixes is necessary. It should be noted, however, that this might requires much more resources since we have to analyze \textit{all} BGP updates during the period, instead of picking up periodic samples.

In our analysis, we use the variable AS path length extensively. However, the length of AS path does not automatically correlate to the performance level experienced by end user. For example, in A-Root case, there are situations where VPs have different IPv4/IPv6 origin ASes, while the paths are identical up to the penultimate AS hop (equal lengths). This strongly indicates that end users reside within those VP networks are directed towards different IPv4 and IPv6 instances, which is very likely to be located in different locations. Thus, the performance of both protocols would be different, even though the path length is the same. Another case is for Root Servers with single origin AS that use only few upstream providers for all of its instances. Even if the IPv4 and IPv6 paths of a certain VP is identical, there is a possibility that IPv4 and IPv6 traffic are still routed towards different instances. AS path length also does not provide us information about cold-potato routing\footnote{the tendency to keeps traffic inside a single AS as long as possible, usually implemented by providers for their customers' traffic} in transiting ASes, which could lead to longer delay. The only way to measure the real performance is by conducting data-plane measurement such as \texttt{traceroute} or special DNS queries. 

In this thesis, the visualization tool is served as proof-of-concept. Currently, it only uses historical data retrieved from RIS. It can be easily extended to dynamically retrieve data directly from RIS to provide near-real-time visualization. It can also be modified to retrieve data from other sources, such as live streaming from RouteViews or from OpenBMP in the future, to provide more comprehensive data. The visualization can also be combined with data-plane measurement and server monitoring to provide complete view of the system. Furthermore, an autonomous monitoring system based on MAPE-K \cite{computing2003architectural} can be developed, so that a change detected in the routing system or server load that exceeds certain threshold may automatically trigger some follow-up action \textit{e.g.}, to boot up new instances in some underserved areas.