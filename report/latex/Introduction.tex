\chapter{Introduction}
\label{ch01:intro}

IP anycast \cite{rfc1546} is a technique to share the same IP address among multiple nodes in multiple locations relying on the routing system to map clients to an anycast node (\textit{one-to-any} connection) based on certain parameters, \textit{e.g.}, server proximity, server load, and so on. It started to gain momentum after the DDoS attack targeting all DNS Root Servers on 2002, causing 9 out of 13 Root Servers to be out of service for a moment. The attack caused the link to be congested, leading to unreachable service experienced by user, even though the servers itself remained fully operational. The use of IP anycast enabled Root Servers operators to mitigate such problem. By spreading their instances around the globe, the DDoS attack can be localized to a certain instance, while instances on the other locations remains functional. Another benefit of anycast is to bring the service closer to the users thus reducing service response time, while at the same time keeping the configuration at user side simple. Today, IP anycast is also employed by other distributed services as well, such as CDN, web hosting, and so on.

Despite of its simplicity, IP anycast is difficult to manage. It is because IP anycast completely depends on the routing system--typically BGP--to select the serving anycast node. BGP itself is well-known for its complexity; mainly because it does not route packets solely based on the shortest path, but also takes into account some other considerations in the form of routing policies. Improper BGP configuration could lead to suboptimal routing, causing worse quality of service. For critical service such as DNS, this is a very important issue, since routing configuration also contributes to the latency experienced by users. DNS is a fundamental Internet protocol where many other protocols are relying on it to operate properly (\textit{e.g.}, mail, web). Slow DNS query results in slow response time to them as well. On another side, the deployment strategy of IPv6 to smoothly replace IPv4 allows the IPv4 and IPv6 coexistence in a network. Ideally both protocols should have similar performances. However, this is not always the case. Study from \cite{7145323} that performed measurements against 100 popular dual-stacked websites shows that the performances are sometimes different. Furthermore, performance over IPv6 paths is comparable to those over IPv4 if the AS-level paths are the same. However, it can be much worse if the AS-level paths differ \cite{Dhamdhere:2012:MDI:2398776.2398832}. It shows that having the knowledge over the global Internet for both IPv4 and IPv6 is important to ensure the anycast service is running similarly.

This thesis assesses the differences between IPv4 and IPv6 service coverage (\textit{catchment areas}) from anycast service. DNS Root Servers is used as the case study, primarily because DNS is the pioneering application that heavily uses anycast. Nevertheless, the methodology used in this thesis can be easily applied to other IP-anycast services as well. Here, this thesis is focused on the control plane aspect, \textit{i.e.}, BGP as the routing system used to deliver packets globally. We obtained data from RIPE RIS project \cite{ripe-ncc-ris} that collects BGP routing information from various locations on the globe. The historical BGP data between the first time Root Servers used IPv6 in the beginning of 2008 and June 1\textsuperscript{st} 2016 is studied. The evolution of IPv4 and IPv6 catchment areas of selected Root Servers over the time period is analyzed and then specifically the differences between them are studied. Finally, a visualization tool to help operator assessing their IPv4/IPv6 catchment areas is also developed.

%This report investigates the current practices and monitoring requirements for (anycast) DNS. The recent studies and works in related fields are investigated in order to get understanding of the state-of-the-art of anycast DNS monitoring activities. The goal of this study is to understand current practices of DNS monitoring and requirements for monitoring anycast DNS, and to represent the data towards operators.

\section{Goals}
\label{ch01:goals}

The goal of this thesis is to assess the differences between IPv4 and IPv6 catchment areas of an anycasted services, with DNS Root Servers as the case study. Therefore, the following main research question (RQ) is used:

\textbf{RQ: How different is IPv4 and IPv6 catchment areas of DNS Root Servers?}

In order to address the main RQ, we define four sub RQs as the following: 

\begin{description}
	\setlength{\itemsep}{1pt}
	\setlength{\parskip}{0pt}
	\item [\textbf{RQ.1}] \textit{\textbf{How can we measure the control plane of anycast DNS system?}} There are several methodologies of anycast measurement found in the literature during the last 15 years. Some relevant measurement projects are also present today. Those are discussed to find the most appropriate one for this thesis.
	\item [\textbf{RQ.2}] \textbf{\textit{How do IPv4 and IPv6 catchment areas evolve over the time?}} With the booming of the Internet, IP networks in general are constantly expanding. Infrastructures are continuously being deployed and network interconnections between organizations are being made. This results in the dynamics of Root Server' anycast networks over the time as well. 
	\item [\textbf{RQ.3}] \textbf{\textit{How different is IPv4 and IPv6 catchment areas?}} IPv6 networks are built years after IPv4 ones, and not as vast as its predecessor yet. It is interesting to find out to what extent the difference is from control-plane perspective.
	\item [\textbf{RQ.4}] \textbf{\textit{How to represent the knowledge to the operator?}} Visualization is the best method to represent the knowledge of the networks, so that the operator may  easily assess the IPv4 and IPv6 catchment areas of their anycast service.
\end{description}

\section{Structure}
\label{ch01:structure}
This thesis is organized as follows. Chapter \ref{ch02} provides related background knowledge of this thesis and state-of-the-art of anycast measurement, especially from control-plane perspective. It provides partial answer for RQ.1. Chapter \ref{ch03} explains about our methodology and considerations used in this thesis. It provides the final part for RQ.1 answer. Chapter \ref{ch04} discusses about the result of this work. It provides the answers for RQ.2, RQ.3, and RQ.4. Finally, Chapter \ref{ch05} concludes this thesis by providing the concluding remarks and future works.